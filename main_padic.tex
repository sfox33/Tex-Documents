\documentclass[a4paper]{article}

\usepackage[english]{babel}
\usepackage[utf8x]{inputenc}
\usepackage{amsmath}
\usepackage{amssymb}
\usepackage{graphicx}
\usepackage[colorinlistoftodos]{todonotes}
\usepackage{notes}
\usepackage{enumitem}
\usepackage{yfonts}
\setlength{\parindent}{0pt}
%\usepackage{indentfirst}

\title{Introduction to the p-adic Numbers}
\author{Sean Fox}

\begin{document}
\maketitle

\begin{abstract}
Let $p$ be any prime number and $x$ be a rational number.  
We denote by $|x|_p$ the highest power of $p$ that divides $x$.  We show
that $|x|_p$ is a norm on $\mb Q$ and that $\mb Q$ can be completed
with respect to this norm to form the p-adic numbers.  We examine the
basic analytic, algebraic, and topological properties and implications 
of the p-adic norm and p-adic numbers.
\end{abstract}
 
\section{Introduction}
\indent In most applications of arithmetic, we write our numbers in base 
10, otherwise known as the decimal system.  This means that any 
integer in base 10 can be represented as a summation
\[
	\sum_{n=0}^{\infty}a_n10^n=a_010^0+a_110+...+a_{n-1}
    10^{n-1}+a_n10^n+...,\;\;\;a_n\in\mb Z_{10}
\]
Furthermore, for any rational number, we can use negative numbers in
the summation to get
\[
	\sum_{n=-\infty}^{\infty}a_n10^n=\dots+a_{-2}10^{-2}+a_{-1}10^{-1}+
    a_010^0+a_110^1+a_210^2+\dots
\]
To give an example, we can expand the integer 532 as such:
\[
	532=5*10^2+3*10^1+2*10^0=5*100+3*10+2*1
\]

In different studies, it can be useful to work in different bases.  
For example, computers make heavy use of the binary system (base 
2) in order to perform computations. In number theory, the set of 
prime numbers are a significant area of study.  Occasionally, it 
has become useful in parts of number theory to represent numbers 
in base $p$, where $p$ is some fixed prime number.

These numbers in base-p are called p-adic numbers, and they were first 
described by Kurt Hensel in 1897 with earlier work by Ernst Kummer 
hinting at their existence when he proved cases of Fermat's Last 
Theorem for prime exponents.  Hensel's motivation for describing the 
p-adic numbers came from the relationship between the integers and the
polynomials with complex coefficients.  It is known by the Fundamental
Theorem of Algebra that any non-constant polynomial of one variable has 
at least one root in the set of complex numbers.  What this means is that
we can write a polynomial as such
\[
	f(x)=a(x-\alpha_1)(x-\alpha_2)\dots(x-\alpha_n)
\]
where $a,\alpha_1,\alpha_2,\dots,\alpha_n\in\mb C$.  Furthermore, we know 
that this decomposition is unique.  If this sounds familiar, it should.  
It is also well know by the Fundamental Theorem of Arithmetic that any 
integer can be written uniquely as a product of prime numbers.  What all of 
this implies is that both the set of integers and set of polynomials of 
single variables are both factorable.

Hensel noticed this relationship, but he also noticed something more. 
Given a particular complex number $\alpha$, we can write a polynomial
$P(X)$ as a Taylor series from Calculus II.
\[
	P(X)=\sum_{i=0}^n a_i(X-\alpha)^i=\alpha_0+\alpha_1(X-\alpha)+
    \alpha_2(X-\alpha)^2+\dots+\alpha_n(X-\alpha)^n
\]
Similarly, as we started to show earlier, integers can be expanded 
depending on a set base.  Since integers are all uniquely factorable by 
prime numbers, we can set up something similar to a Taylor series 
expansion for a fixed prime number p and an integer $m$.
\[
	m=\sum_{i=0}^na_ip^i=a_0+a_1p+a_2p^2+\dots+a_np^n,\;\;\;a_i\in\mb Z
    \text{ and }0\leq a_i\leq p-1
\]

We recall from Calculus II that a Taylor series has a radius of
convergence.  That is, there exists some number $R$ such that our 
expansion will only converge to a number if $|x-\alpha| < R$.  This means 
that a Taylor series gives us information only in a local area around
$\alpha$.  For example, we know that if $X=\alpha$, then our polynomial
will vanish.

It turns out that a p-adic expansion (expanding an integer in base $p$)
also gives us information local to our prime number $p$.  But rather than
a vanishing point, it tells us the divisibility of a number up to some
order of $p$.  For example, let's expand the number $72$ with $p=3$.
\[
	72=0+0*3+2*3^2+2*3^3
\]

From looking at the expansion, we can easily see that $72$ is divisible by
$3^2=9$.  Furthermore, it turns out that we can use the same line of
reasoning for the rational numbers since they are the field of fractions
of the integers - in other words, since any rational number can be
represented as a fraction whose numerator and denominator are both
integers.  It was shown earlier that a rational number can be expanded if
we include negative numbers for our index.  So instead of the Taylor
series expansion, we can compare the expansion of a rational number to a
Laurent series from Intro to Complex Analysis.

It turns out that these numbers have found their place in many applications
since their discovery.  The p-adic numbers have their own type of calculus 
due to a special absolute value, they are currently being studied 
alongside elliptic curves in the area of cryptography, and they even have
their own fields of study in quantum physics and complex dynamics.  But
one of the most basic uses is in number theory in which they allow us to 
easily solve equations modulo every power of a prime number $p$.

However, before we jump into uses of p-adic numbers, we need to have a few
questions answered.  For example, we know when a Taylor 
series and a Laurent series converge.  But how do we know when the p-adic 
expansion of a number converges? When we write the p-adic expansion of a
number, it makes sense that the sum of numbers in base p should converge if
they are going to equal a number.  The result is that, in p-adic
numbers, $p^n$ gets smaller as $n$ grows.  We will show why this is, but
first, we have to talk about how we will measure the size of a p-adic
number - with absolute values.

\section{Absolute Values}

\subsection{Real Absolute Value}
Before we dive into absolute values and valuations, we remind 
ourselves of the definition of a ring from Intro to Modern Algebra.

\begin{definition}[Rings]
  A \(\bf{ring}\) is a set of numbers R under two operations,
  usually denoted as + (addition) and * (multiplication), such that, for
  $a,b,c\in R$ it fulfills the following properties:
  \begin{enumerate}[series=ring]
	\item $a+b\in R$
    \item $a*b\in R$
    \item $(a+b)+c=a+(b+c)$
    \item $(a*b)*c=a*(b*c)$
    \item $a+b=b+a$
    \item There exists an element $0_R\in R$ such that $a+0_R=a=0_R+a$ for 
    	all $a\in R$.
    \item There exists an element $-a\in R$ such that $a+(-a)=0_R=(-a)+a$
    	for all $a\in R$.
    \item $a(b+c)=ab+ac$ and $(b+c)a=ba+ca$
  \end{enumerate}
  Furthermore, R is a \(\bf{commutative\;ring}\) if 
  \begin{enumerate}[resume=ring]
	\item $a*b=b*a$ for all $a,b\in R$
  \end{enumerate}
  R is a $\bf{ring\;with\;identity}$ if there exists an element $1_R\in R$ called 
  the identity such that
  \begin{enumerate}[resume=ring]
    \item $a*1_R=a=1_R*a\;\;\;\forall\;a\in R$
  \end{enumerate}
  R is an $\bf{integral\;domain}$ if it is a commutative ring with identity 
  $1_R\neq 0_R$ that contains no zero divisors.  That is,
  \begin{enumerate}[resume=ring]
    \item $a*b=0_R\Rightarrow a=0_R\text{ or }b=0_R$
  \end{enumerate}
  And finally, R is a \(\bf{field}\) if it is an integral domain such that, 
  \begin{enumerate}[resume=ring]
    \item For all $a\in R$, there exists $a^{-1}\in R$ where 
    $a*a^{-1}=1_R=a^{-1}*a$.
  \end{enumerate}
\end{definition}

We will be dealing more with fields in depth after we formally construct
the p-adic numbers.  For now, we wanted to remind ourselves of the
definition of a ring so that we can fully understand the definitions to
come throughout this section.  To actually understand how we will
construct the p-adic numbers, we will have to go back to our years in
middle school and generalize one of the more basic concepts of algebra.

\begin{definition}[Absolute Values]
  Let K be a field and \(a,b\in K\).  Then an 
  \(\bf{absolute\;value}\) (also called a $\bf{norm}$) is a mapping 
  \(|\cdot|:K\to\mb R_+\) with the properties
  \begin{enumerate}
    \item \(|a|=0\), if and only if \(a=0_K\)
    \item \(|ab|=|a|*|b|\) for all \(a,b\in K\)
    \item \(|a+b|\leq |a|+|b|\)
  \end{enumerate}
  Specifically, $|\cdot|$ is called an 
  \(\bf{archimedian\;absolute\;value}\).
  Instead of property 3 (known as the Triangle Inequality), if 
  \(|\cdot|\) has the property that 
  \[
  	3'.\;\;\;\; |a\pm b|\leq\max\{|a|,|b|\}
  \]
  which is known as the Strong Triangle Inequality, then 
  \(|\cdot|\) is \(\bf{non-archimedian}\).
\end{definition}

The absolute value that we are probably most familiar with is defined 
below:

\[
  |x|=
  \begin{cases} 
      x & x\geq 0 \\
      -x & x<0 
   \end{cases}\;\;\;\;\;
   \forall\;x\in K
\]

This absolute value is used often, and it has several names.  It has been
called the real absolute value and the euclidean absolute value or euclidean
norm.  However, we are trying to generalize this concept and
will be introducing several different types of absolute values.  Because
of this, it is common to distinguish this function in p-adic analysis and
valuation theory by \(|\cdot|_\infty\) and call it the absolute value at
infinity.  The reason for this is mostly notational, but it will become clear
later.  For the purposes of this paper, we will refer to it as the real
absolute value.

\begin{properties}[Real Absolute Value]
Written below are some well-known properties of the real absolute value for the
sake of later comparison.
  \begin{enumerate}
    \item \(|0|_\infty=0\)
    \item \(|a|_\infty=|-a|_\infty=a>0\;\;\forall a\in\mb R\)
    \item \(|a\pm b|_\infty\leq |a|_\infty+|b|_\infty\)
    \item \(|ab|_\infty=|a|_\infty*|b|_\infty\)
  \end{enumerate}
\end{properties}

By comparing the real absolute value to our definition of a general
absolute value, we can see that the real absolute value is an
archimedean absolute value.

The real absolute value is often used to denote distances 
and magnitudes between numbers.  If applied to a single number, the
real absolute value gives the distance of the number from 0.  
It does not take much to extend
that idea to other absolute values as well.  For example, if we extend
the real absolute value from the real numbers to the
complex numbers, which is referred to as the modulus,
we get the distance of a number from the origin.

\begin{figure}[!h]
  \centering
  \includegraphics[width=0.3\textwidth]{modulus.jpg}
  \caption{\label{fig:modulus}The modulus of the complex number z
  represented by \(|z|\).}
\end{figure}

To further our understanding of absolute values, we will cover the most
basic of absolute values: the trivial absolute value.

\begin{definition}[Trivial Absolute Value]
Let \(K\) be a field.  \(|\cdot|:K\to\mb R\) is the trivial 
absolute value defined by
\[
  |x|=
  \begin{cases} 
      1 & x\neq 0 \\
      0 & x=0 
   \end{cases}\;\;\;\;\;
   \forall\;x\in K
\]
\end{definition}

The real absolute value and the modulus are both examples
of archimedean absolute values.  That is, they only satisfy the triangle 
inequality. The trivial absolute value, however, is a non-archimedean
absolute value because it satisfies the strong triangle inequality as
shown below.

\begin{theorem}
  The trivial absolute value is non-archimedean.
\end{theorem}
\begin{proof}
  Let \(K\) be any field, \(x,y\in K\), and let \(|\cdot|\) be the 
  trivial absolute value.
  \begin{enumerate}
  \item Suppose that \(x,y\neq0_K\), the additive identity of K. 
  If \(x\pm y\neq 0_K\), it follows that
  \[
  	|x\pm y|=1\Rightarrow |x\pm y|=\max\{|x|,|y|\}
  \]
  \item Suppose that \(x,y\neq0_K\).
  If \(x\pm y=0_K\), it follows that
  \[
  	x\pm y=0_K\Rightarrow 0=|x\pm y|<\max\{|x|,|y|\}=1
  \]
  \item Suppose that \(x,y=0_K\).  It follows that
  \[
  	|x\pm y|=0=\max\{|x|,|y|\}
  \]
  Therefore, for all \(x,y\in K\), \(|x\pm y|\leq\max\{|x|,|y|\}\)
  which means the trivial absolute value satisfies the strong
  archimedean equality and is a non-archemedian absolute value.
\end{enumerate}
\end{proof}

As a side note, one other interesting fact about the trivial absolute
value is that it is the only absolute value that can be applied to a
finite field.

\begin{theorem}
  The only absolute value on a field with finitely many elements
  is the trivial absolute value.
\end{theorem}
\begin{proof}
  Let \(K\) be a finite field.  As \(K\) is a field, \(K\) 
  contains the identity elements \(0_K,1_K\).  Let \(|\cdot|\) be
  an arbitrary absolute value on \(K\).  Clearly by definition of
  absolute value, \(|0_k|=0\).  Since \(1_K\) is the
  multiplicative identity, \(1_K=1_K*1_K\Rightarrow|1_K|=
  |1_K*1_K|\leq|1_k|*|1_K|\).  Since \(|1_K|>0\), it follows
  that \(|1_k|=1\).  Now let \(x\in K,\;x\neq0_K\).  Because \(K\)
  is finite, there exists some integer \(n\) such that 
  \(x^n=x\) (we can let n be the number of elements in K).  By 
  taking the absolute value of both sides, we have 
  \(|x^n|=|x|^n=|x|\).  Since \(|x|>0\), then the only solution is
  that \(|x|^n=|x|=1\).  Therefore, \(|\cdot|\) on a finite field
  \(K\) can only be the trivial absolute value.
\end{proof}

\subsection{P-adic Absoute Value}

The trivial absolute value is not the only non-archimedean
absolute value.  There is one more significant non-archimedean
absolute value that we will discuss that will be the 
foundation of building the field of p-adic numbers.  However, we
must first define the p-adic valuation.

\begin{definition}[Valuation]
A $\bf{valuation}$ on a field $K$ is a real-valued function on
$K\backslash\{0\}$ satisfying:
\begin{enumerate}
  \item $v(xy)=v(x)+v(y)$
  \item $v(x+y)\geq\min\{v(x),v(y)\}$
\end{enumerate}
\end{definition}

\begin{definition}[p-adic Valuation]
  Let $p$ be a fixed prime in $\mb Z$. The 
  \(\bf{p-adic\;\; valuation}\) on \(\mb Z\) is the function 
  \(v_p:\mb Z\backslash\{0\}\to\mb R\) defined as follows: for
  each nonzero \(n\in\mb Z\), let $v_p$ be the unique positive
  integer satisfying
  \[
  	n=p^{v_p(n)}n'\;\;
    \text{ with }p\nmid n'
  \]
  Furthermore, we extend $v_p$ to the field of rational numbers as 
  such: if \(\frac{a}{b}=x\in\mb Q^x\), then
  \[
    v_p(x)=v_p(a)-v_p(b)
  \]
\end{definition}

By convention, we let \(v_p(0)=+\infty\).  We can see why if we 
plug in the values into our definition.
\[
	0=p^{v_p(0)}n'\Rightarrow p^{v_p(0)}\big|0\Rightarrow
    p^\infty|0
\]
As we can see, any prime number will divide 0 infinitely many 
times.  This choice will also make a lot of things easier when we 
form the p-adic absolute value.  Furthermore, the definition of the p-adic valuation can be 
extended to the rational numbers such that for \(x\in\mb Q^x\) by
\(x=p^{v_p(x)}\frac{a}{b}\) where \(p\nmid ab\).  We also have the following trick to help when computing the p-adic valuation of a rational number.

\begin{theorem}
  For any \(x\in\mb Q^x\), the value of \(v_p(x)\) does not depend
  on the representation of \(x\).  In other words, if 
  \(x=\frac{a}{b}=\frac{c}{d}\), then 
  \(v_p(a)-v_p(b)=v_p(c)-v_p(d)\).
\end{theorem}

  To help make all of this clear, we will 
  compute \(v_5(400)\) and \(v_3(\frac{123}{48})\).
\begin{enumerate}[label=\emph{\alph*})]
  \item \(v_5(400):\;\;400=5^{v_5(400)}*n\Rightarrow 
    5^{v_5(400)} = \frac{400}{n}\Rightarrow 5^2=\frac{400}{16}
    \Rightarrow v_5(400)=2\)
  \item \(v_3\Big(\frac{123}{48}\Big)=
    v_3\Big(\frac{41}{16}\Big)=v_3(41)-v_3(16)\\
    41=3^{v_3(41)}n_1\Rightarrow 41=3^{0}*41\Rightarrow
    v_3(41)=0\\
    \;\;\;16=3^{v_3(16)}n_2=3^0*16\Rightarrow v_3(16)=0\\
    v_3\Big(\frac{41}{16}\Big)=0-0=0\)
\end{enumerate}

It turns out that valuations and absolute values are very 
similar.  They both are generalized functions that allow
us to measure various things.  With absolute values, it 
turns out be distance.  For valuations, it turns out to be
divisibility and multiplicity.  But if we look closer, the similarities are what we will use to construct an absolute value for the p-adic numbers.

After looking at the above examples and the definition for the 
p-adic valuation, it is clear that another way to think of the 
p-adic valuation is that it answers the question of how divisible 
a number $x$ is by a prime number $p$.  A large value for \(v_p(x)\)
means that \(x\) is highly divisible by \(p\), and the opposite 
is true for small values of \(v_p(x)\).  In a way, the p-adic
valuation measures a different kind of magnitude.

As the p-adic valuation is a valuation, it satisfies the properties that
are in the valuation definition.  We wish to now compare those properties
to 2. and 3. in the definition of 
absolute value. They are listed below:

\begin{center}
  \begin{tabular}{|c|c|}
    \hline
      Valuation & Absolute Value\\
    \hline
      $1.\;\;\;v(xy)=v(x)+v(y)$ & $2.\;\;\;|xy|=|x|*|y|$\\
      $2.\;\;\;v(x+y)\geq\min\{v(x),v(y)\}$ & $3.\;\;|x+y|\leq |x|+|y|$\\
    \hline
  \end{tabular}
\end{center}

Since the p-adic valuation measures a magnitude of
divisibility by a prime number $p$, we can make our valuation act as
an absolute value by changing the direction of the sign in the second
valuation property.  This is done by multiplying both sides of the
inequality by $-1$. Also, if we make our valuation an exponent to some base
number, we can fulfill the first property of a valuation and the second
property of an absolute value.  After manipulating the definition
for the p-adic valuation with these in mind, we get the p-adic
absolute value.

\begin{definition}[p-adic Absolute Value]
  For any \(x\in\mb Q\), we define the $\bf{p-adic\;absolute\;value}$
  or $\bf{p-adic\;norm}$ of x by
  \[
  	|x|_p=
    \begin{cases}
      p^{-v_p(x)} & x\neq 0\\
      0 & x=0
    \end{cases}
  \]
\end{definition}

The p-adic absolute value has an interesting definition, so there
are a few things worth noting.  First, we can finally see why we
made sure to set the convention that \(v_p(0)=+\infty\).  When
this is the case, 

\[
	|0|_p=p^{-v_p(0)}=p^{-(+\infty)}=\frac{1}{p^\infty}=0
\]

Thus, we have a reason to say that \(|0|_p=0\) which follows from our 
definition of an absolute value.  It is also worth 
nothing that, for integers, \(|x|_p\leq1\) for all $x\in\mb Z$. 
This may not necessarily be the case for rational numbers, but we
will talk about these results later.

As we hinted at earlier, the p-adic absolute value is non-archimedean. 
This has some unique properties when actually
applied to different numbers, but this may be easier to appreciate
if we first show some examples of how to take the p-adic absolute
value of some numbers.  This will become a lot easier, and more
compact to write, if we make use of the following theorem from Number 
Theory.

\begin{theorem}[The Fundamental Theorem of Arithmetic]
Every integer greater than 1 is either prime itself or can be
uniquely, up the order of factors, represented as a product of
primes.
\end{theorem}

For example, \(24=2^3*3\) and \(39=3*13\).  We can also see that
any rational number can be expressed as a product of primes if it
is represented as a fraction of integers.  Observe that 
\(\frac{39}{8}=3*13*2^{-3}\).  With this in mind, we can decompose any
rational number into its prime factorization to find the p-adic
absolute value for any prime number p.

Consider the number \(\frac{63}{550}\).  It's prime factorization
can be written as \(\frac{63}{550}=2^{-1}*3^2*5^{-2}*7*11^{-1}\).  We
then write the p-adic absolute value to be
\[
	\Big|\frac{63}{550}\Big|_p=
    \begin{cases}
      2 & p=2\\
      \frac{1}{9} & p=3\\
      25 & p=5\\
      \frac{1}{7} & p=7\\
      11 & p=11\\
      1 & \text{All other primes}
	\end{cases}
\]

Normally $p$ would be a fixed prime, but we considered all $p$'s
in the above example not only to show several examples, but to
also notice a pattern.  The number $\frac{63}{550}$ is very divisible 
by $3$ and $7$ if we look at the prime factorization.  We can also see 
that our number is not very divisible by $2$, $5$, or $11$.  But if we 
look at out p-adic absolute values, we can see that the more divisible 
that a number is by a prime $p$, the smaller its p-adic absolute 
value will be.  Our smallest value ends up being the 3-adic absolute 
value while the largest one is the 5-adic absolute value.

This time, let's compare the p-adic absolute value for two numbers
with a fixed $p$, say $p=3$.  Let's compare the numbers 82 and
1.  If we use the real absolute value to analyze the 
difference, we see that \(|82-1|_\infty=81\) which tells us that 
the two numbers are far apart on the real number line.  
However, if we use the p-adic absolute value with $p=3$, we get 

\[
	|82-1|_3=|81|_3=3^{-v_3(81)}=3^{-4}=\frac{1}{81}
\]

If we use the p-adic absolute value to denote distance as we do
with the real absolute value, we can see that 1 and 82, which 
are far apart in the real numbers, are close together in the 3-adic
number system because their difference, $81$, is divisible by $3$.
We will be dealing more with these implications when we discuss
metrics.

\section{Completing $\mb Q$}

\subsection{The Real Numbers}

Now that we have introduced the p-adic absolute value, we have answered
our question of convergence.  Because of the p-adic absolute value, we
can see that, in a p-adic number system, $p^n$ gets smaller as $n$
increases. As a result, we can finally start to construct the field of
p-adic numbers.  Formally, this field, denoted $\mb Q_p$, can be thought
of as an alternate completion of the rational numbers to the real
numbers.  

\begin{remark}
Throughout this paper, we we will write $\mb Q_p$ for the set of p-adic
numbers and $\mb Z_p$ for the set of p-adic integers - not to be confused
with the set of integers modulo p, which we write as the quotient ring 
$\mb Z/p\mb Z$.  All of these will be defined formally later, but their
symbols will be used throughout the next section.
\end{remark}

To start the process of constructing completions of the rational
numbers, we have to first introduce a few definitions.

\begin{definition}[Completion]
A field $K$, or more generally a metric space, is called 
$\bf{complete}$ with respect to an absolute value $|\cdot|$ if
every Cauchy sequence of elements in $K$ has a limit.
\end{definition}

\begin{definition}[Cauchy Sequence]
A sequence of elements $x_n$ in a field $K$ is called a 
$\bf{Cauchy\;\;sequence}$ if for every $\epsilon>0$ one can find a
bound $M$ such that we have $|x_n-x_m|<\epsilon$ whenever $n,m\geq
M$.
\end{definition}

The general idea is that every Cauchy sequence of elements of a
completed field should have a limit inside of that field.  If we
measure the distances according to the real absolute value
between every two consecutive elements of a Cauchy sequence, the 
distance should gradually get less and less to the point where the 
distance is almost negligible.  So reaching a limit makes sense.  
However, it should be easy then to see that \(\mb Q\) is not complete.

Consider the sequence of rational numbers 
\(\{3,3.1,3.14,3.141,3.1415,3.14159,...\}\). 
It should be clear that this sequence is Cauchy as the distance
between two subsequent elements gets smaller and smaller.  
However, our sequence approaches \(\pi\) which is not a rational 
number.  A similar sequence can be made for \(\sqrt{2}\): 
\(\{1,1.4,1.41,1.414,1.4142,1.41421,...\}\).  The elements are all 
rational numbers, but the limit of the sequence is not.

This is the idea behind one way of defining the set of real 
numbers - by completing \(\mb Q\) with respect to 
\(|\cdot|_\infty\) by forming a union of \(\mb Q\) and the limits 
of every Cauchy sequence in \(\mb Q\).  But what if we changed up
the process slightly?  Instead of completing \(\mb Q\) with 
respect to \(|\cdot|_\infty\), an archimedean absolute value, what 
if we completed \(\mb Q\) with respect to our non-archimedean 
p-adic absolute value?

\subsection{Forming $\mb Q_p$}

We have now seen that $\mb Q$ is not complete with respect to 
$|\cdot|_\infty$; and, while it will not be shown here, $\mb Q$ is not 
complete with respect to the p-adic absolute value either. It turns 
out that we can actually form another completion by a similar method to
how we formed the real numbers but by using the p-adic absolute value.
However, there is something to consider before we start forming Cauchy
sequences.  With the real absolute value, it is easier for us to understand
a sequence of numbers that gets closer and closer together, but it is
important to realize that the following does not imply a sequence is 
Cauchy:

\[
	\lim_{n\to\infty}|x_{n+1}-x_n|_\infty=0
\]

One would think that it would make perfect sense for a sequence
whose distances between subsequent terms decreases to 0 to be
Cauchy and therefore convergent.  However, we recall that the
harmonic series from Calculus II is an example of such a series that is
divergent.

\begin{theorem}
The $\bf{harmonic\;series}$ is defined by 
\[
	\sum_{n=1}^\infty\frac{1}{n}=1+\frac{1}{2}+\frac{1}{3}+
    \frac{1}{4}+\frac{1}{5}+...
\]
The harmonic series diverges.
\end{theorem}

This is interesting because the distance between any two 
subsequent terms of the sequence $\\$($|x_{n+1}-x_n|_\infty$) will 
get smaller and smaller as $n$ tends to infinity.  However, the
series of that sequence will never converge.  Since the distance between 
every pair of subsequent terms of the sequence \(\{\frac{1}{n}\}\) 
will always increase the sum of its series, \(\{\frac{1}{n}\}\) cannot be
Cauchy.  One fun fact about non-archimedean norms
though, is that this type of sequence will always be Cauchy under
a non-archimedean norm.

\begin{lemma}
A sequence $\{x_n\}$ of rational numbers is a Cauchy sequence with
respect to a non-archimedean norm $|\cdot|$ if and only if
\[
	\lim_{x\to\infty}|x_{n+1}-x_n|=0
\]
\end{lemma}
\begin{proof}
If $m=n+r>n$, we get by the properties of a non-archimedean norm 
that
\[
\begin{gathered}
  |x_m-x_n|=|x_{n+r}-x_{n+r-1}+x_{n+r-1}-x_{n+r-2}+
  \dots+x_{n+1}-x_n|\\
  \leq\max\{|x_{n+r}-x_{n+r-1}|,
  |x_{n+r-1}-x_{n+r-2}|,\dots,|x_{n+1}-x_n|\}
\end{gathered}
\]
It follows as a result that the lemma must hold true.
\end{proof}

The result of all of this being that it is easier to construct Cauchy
sequences with \(|\cdot|_p\) than it is with \(|\cdot|_\infty\).  
Again, our goal is to complete $\mb Q$ with respect to \(|\cdot|_p\) by
adding the limits of all of the Cauchy sequences in $\mb Q$ formed by 
\(|\cdot|_p\) to \(\mb Q\).  However, as we are creating a field
from the perspective of \(\mb Q\), it does not make sense to
create numbers that do not exist (the limits).  Rather than make up
numbers, we will instead use the Cauchy sequences themselves.

\begin{definition}
Let $|\cdot|_p$ by the non-archimedean p-adic norm on $\mb Q$.  We
denote by \(\mathcal{C}_p(\mb Q)\) the set of all Cauchy Sequences
of rational numbers:
\[
	\mathcal{C}_p(\mb Q):=\Big\{\{x_n\}:\{x_n\}\text{ is a Cauchy
    sequence with respect to }|\cdot|_p\Big\}
\]
\end{definition}

Furthermore, we can see that \(\mathcal{C}_p(\mb Q)\) is a ring.  
Addition and multiplication in \(\mathcal{C}_p(\mb Q)\) is closed and
defined below.
\[
	\begin{gathered}
      \{x_n\}+\{y_n\}=\{x_n+y_n\}\\
      \{x_n\}*\{y_n\}=\{x_n*y_n\}
    \end{gathered}
\]
The zero element is \(\{0,0,0,0,...\}\) and the identity element 
is \(\{1,1,1,1,...\}\). Since these sequences are formed of
rational numbers, it follows that \(\mathcal{C}_p(\mb Q)\) is a
commutative ring.

But we do have to make sure of something.  We just formed a ring
that we claim is about to extend the rational numbers.  So before
we move forward, we should make sure that 
\(\mb Q\hookrightarrow \mathcal{C}_p(\mb Q)\), that the rational
numbers are embedded in our ring.  Well, this is actually easy to
show based on our recently proven lemma.  If construct a constant 
sequence - that is, for 
$x\in\mb Q$, we form $\{x_n\}=\{x,x,x,x,\dots\}$ - 
then this sequence will be Cauchy under $|\cdot|_p$ because
$\lim_{n\to\infty}|x_{n+1}-x_n|_p=\lim_{n\to\infty}|0|_p=0$.  Thus, 
\(\mb Q\hookrightarrow \mathcal{C}_p(\mb Q)\).

We are not quite done yet.  It turns out that
we can create several Cauchy sequences that converge to the same
number.  While, for our purposes, these sequences are essentially 
the same (we care about their limits, not necessarily the
sequences), they are still different objects in 
\(\mathcal{C}_p(\mb Q)\).  In order to create an equivalence for 
these sequences that converge to the same number, we remind 
ourselves of some ring theory concepts from Intro to Modern Algebra.

\begin{definition}[Ideal Ring]
An $\bf{ideal}$ $I$ of a ring $R$ is a subring such that for
any element $r\in R$ and any element $i\in I$, $i*r\in I$ and 
$r*i\in I$.  Furthermore, we say that two elements $a,b\in R$ are 
$\bf{congruent\;modulo\;I}$, written $a\equiv b\pmod I$, provided that
$a-b\in I$.
\end{definition}

\begin{definition}[Maximal Ideal]
A $\bf{maximal\;ideal}$ $M$ is an ideal of a ring $R$ such that for 
any ideal $I\in R$, we know that $I\subset M$ and $M\neq R$.
\end{definition}

\begin{definition}[Quotient Ring]
  Let $R$ be a ring and $I$ be an ideal in $R$.  A $\bf{quotient\;ring}$, 
  written $R/I$ is a ring equipped with a congruence defined by the ideal 
  whose elements are the cosets $a+I$.
\end{definition}

\begin{definition}
We define $\mathcal{N}\subset\mathcal{C}_p(\mb Q)$ to be the ideal
\[
  \mathcal{N}=\Big\{\{x_n\}:x_n\to 0\Big\}=
  \Big\{\{x_n\}:\lim_{x\to\infty}|x_n|_p\to 0\Big\}
\]
of sequences that tend to 0 with respect to the p-adic norm. 
Furthermore, $\mathcal{N}$ is a maximal ideal of 
$\mathcal{C}_p(\mb Q)$
\end{definition}

We have just created a subring of $\mathcal{C}_p(\mb
Q)$ in which the Cauchy sequences that converge to zero
under the p-adic norm are equivalent objects.  This might seem
pointless until we realize that the difference between two Cauchy
sequences that converge to the same number will
converge to zero.  So really, two sequences in $\mathcal{C}_p(\mb Q)$ 
are equivalent with respect to $\mathcal{N}$ if they converge to the 
same number.

With our new ideal, we now have the ability to separate the Cauchy
sequences in $\mathcal{C}_p(\mb Q)$ by sequences that converge to
the same number.  We do this by taking the quotient ring of
$\mathcal{C}_p(\mb Q)$ by $\mathcal{N}$.  Additionally, one property of
quotient rings is that taking a quotient ring by a maximal ideal
results in a field.

\begin{definition}[$\mb Q_p$]
We define the $\bf{p-adic\;numbers}$ to be the quotient of the ring
$\mathcal{C}_p(\mb Q)$ by its maximal ideal $\mathcal{N}$:
\[
	\mb Q_p=\mathcal{C}_p(\mb Q)/\mathcal{N}
\]
\end{definition}

It is important to notice that any constant sequence in
$\mathcal{C}_p(\mb Q)$ will not differ by an element of
$\mathcal{N}$.  In other words, every constant sequence has its
own equivalence class in \(\mb Q_p\).  Thus, 
\(\mb Q\hookrightarrow\mb Q_p\).  All that remains now it prove
that \(\mb Q_p\) is actually a complete space.  In order to that, however,
we must first see that $\mb Q$ is dense in $\mb Q_p$.

\begin{definition}[Open Ball]
Let $K$ be a field with an absolute value $|\cdot|$.  Let $a\in K$ and 
$r\in\mb R_+$.  The \(\bf{open\;ball}\) of radius $r$ and center $a$ is the set
\[
  B_r(a)=\{x\in K:d(x,a)<r\}=\{x\in K:|x-a|<r\}
\]
\end{definition}

\begin{definition}[Dense Subset]
Let $K$ be a field and $|\cdot|$ be an absolute value on that field.  
A subset $S\subset K$ is called $\bf{dense}$ in K if every open ball around
every element of $K$ contains an element of S; in symbols, if for 
every $x\in k$ and every $\epsilon > 0$ we have
\[
	B_\epsilon(x)\cap S\neq\emptyset
\]
\end{definition}

\begin{proposition}
The image of $\mb Q$ in the inclusion $\mb Q\hookrightarrow\mb Q_p$ is a
dense subset of $\mb Q_p$
\end{proposition}
\begin{proof}
Let $\lambda\in\mb Q_p$.  By the construction of $\mb Q_p$, we know that
there exists a sequence $\{x_n\}$ representing $\lambda$.  For each 
fixed positive $m\in\mb N$, we denote the constant sequence $\{x_m\}$. 
In other words, $\{x_m\}$ is a rational number in p-adic form where
$x_m$ is some number from the sequence $\{x_n\}$.  Thus, we
write the sequence $\{x_n-x_m\}_{n=1}^\infty$.  We see that this is
equivalent to writing $\lambda-\{x_m\}$.  Since $\{x_n\}$ is Cauchy, we
see that
\[
	\lim_{n\to\infty}|\lambda-\{x_m\}|_p=
    \lim_{n\to\infty}|x_n-x_m|_p=0
\]
Therefore, it follows that every p-adic number is sufficiently close
to a rational number.  It follows then that $\mb Q$ is dense in 
$\mb Q_p$.
\end{proof}

Before we prove that $\mb Q_p$ is complete with respect to the p-adic
absolute value, we have to settle the slightly confusing issue of
convergence of sequences in $\mb Q_p$.  Remember that $\mb Q_p$ is a
quotient ring of Cauchy sequences.  So a sequence of elements in $\mb
Q_p$ is a sequence of sequences.

\begin{definition}
If $\lambda\in\mb Q_p$ is an element of $\mb Q_p$, and $\{x_n\}$ is 
any Cauchy sequence representing $\lambda$, we define
\[
	|\lambda|_p=\lim_{n\to\infty}|x_n|_p
\]
\end{definition}

\begin{proposition}
$\mb Q_p$ is complete with respect to $|\cdot|_p$.
\end{proposition}
\begin{proof}
Let $\{\lambda_n\}$ be a Cauchy sequence of elements of $\mb Q_p$.  If
$\mb Q_p$ is complete, then this Cauchy sequence must converge.
From the fact that $\mb Q$ is dense in $\mb Q_p$, we know 
that for every $\lambda_n$ we can choose a rational number $a_n$ to
create a constant sequence such that
\[
	|\lambda_n-\{a_n\}|_p=
    \lim_{n\to\infty}|\lambda_n-\{a_n\}|_p<\frac{1}{n}
\]
From the proof of the density of $\mb Q$ in $\mb Q_p$, we can easily
see $\{\lambda_n-a_n\}$ is a Cauchy sequence in $\mb Q_p$. Furthermore,
we see that
\[
	\{a_n\}=\{\lambda_n\}-\{\lambda_n-a_n\}
\]
and as a result $\{a_n\}$ is a Cauchy sequence.  However, we recall 
that $\{a_n\}$ is a constant sequence where $a_n\in\mb Q$.  We will 
denote its representation in $\mb Q_p$ as $\widehat{a_n}$.  We want to 
show that $\widehat{a_n}=\lim_{n\to\infty}\{\lambda_n\}$.  We start by 
seeing that
\[
	\{\widehat{a_n}-\lambda_n\}=\{\widehat{a_n}-\{a_n\}\}-
    \{\lambda_n-\{a_n\}\}
\]
However, we have previously shown that both sequences on the right will 
converge to 0.  This implies that
\[
	\lim_{n\to\infty}|\widehat{a_n}-\lambda_n|_p=0
\]
However, as $\widehat{a_n}$ is a constant number, then it stands to 
reason that 
\[
	\lim_{n\to\infty}\lambda_n=\widehat{a_n}
\]
What this states is that any Cauchy sequence in $\mb Q_p$ will have a 
limit in $\mb Q_p$.  Therefore, $\mb Q_p$ is a complete space.
\end{proof}

\subsection{Otrowski's Theorem and the Product Formula}

Before we start looking at properties of the p-adic number, there 
is an important question that we would like to answer.  Why would
swapping \(|\cdot|_\infty\) with \(|\cdot|_p\) be significant?  Are 
there not more types of absolute values that we could construct in 
order to form new extensions of the rational numbers?  Well, due to an
important theorem, it turns out that every absolute value on \(\mb Q\)
is equivalent to one of these two.

\begin{definition}
Two absolute values \(|\cdot|_1\) and \(|\cdot|_2\) are considered
\(\bf{equivalent}\) if they define the same topology on \(K\),
that is, every open set with respect to one is open with respect
to the other.
\end{definition}

This definition may be easy to say, but it can be a lot harder to 
show.  To help prove equivalence of absolute values, we will make
use of the following theorem.

\begin{theorem}
Let \(|\cdot|_1\) and \(|\cdot|_2\) be equivalent absolute values 
on a field \(K\).  Then there is a positive real number $c$ such 
that \(|\cdot|_1=|\cdot|_2^c\).
\end{theorem}

\begin{theorem}[Ostrowski]
Every non-trivial absolute value on \(\mb Q\) is equivalent to one of
the absolute values \(|\cdot|_p\), where p is a prime number or 
\(p=\infty\).
\end{theorem}
\begin{proof}
Assume that \(|\cdot|\) is non-trivial.  Then we will examine the 
cases where \(|\cdot|\) is archimedean and non-archimedean.

$\bf{Case\;I}$  Suppose that \(|\cdot|\) is archimedean. Our goal is to
show that \(|\cdot|\) is equivalent to \(|\cdot|_\infty\). Since 
$1<|2|\leq|1|+|1|=2$, there is a number $c\in\mb R$, $0<c\leq 1$ for
which
\[
	|2|=2^c
\]
For $n\in\mb N$, we shall prove that
\[
	|n|=n^c
\]
Thus, let $n\in\mb N$, $n\geq 2$, and write $n$ using the base 2
notation
\[
	n=a_0+a_1*2+\dots+a_s*2^s,\;\;\;a_0,a_1,\dots
    \in\{0,1\},\;\;a_s=1
\]
Then $2^s\leq n<2^{s+1}$ so that
\[
	2^{sc}\leq n^c<2^{c(s+1)}\;\;\;\;\;(*)
\]
We first prove that $|n|\leq|n|^c$ as follows.  Applying (*) we 
get
\[
  |n|\leq\sum_{i=0}^s|a_i|\:|2|^i\leq\sum_{i=0}^s2^{ic}=
  2^{sc}(1+2^{-c}+\dots+2^{-sc})\leq n^cM
\]
where $M:=\sum_{i=0}^\infty2^{-ic}$ does not depend on $n$.  Since 
$n$ was arbitrary, we have also
\[
  |n^k|\leq n^{kc}M,\;\;\;k\in\mb N
\]
so that
\[
	|n|\leq \lim_{k\to\infty}n^c\sqrt[k]{M}=n^c\;\;\;\;\;(**)
\]
To prove the opposite inequality, observe that 
\[
	|n|=|2^{s+1}-(2^{s+1}-n)|\geq |2^{s+1}|-|2^{s+1}-n|
\]
Now $|2^{s+1}|=|2|^{s+1}=2^{c(s+1)}$.  By (*) and (**),
\[
	|2^{s+1}-n|\leq(s^{s+1}-n)^c\leq (s^{s+1}-2^s)^c=2^{sc}
\]
so that
\[
	|n|\geq 2^{c(s+1)}-2^{cs}=2^{c(s+1)}(1-2^{-c})
\]
Again, by (*), $2^{c(s+1)}>n^c$; with $M':=1-2^{-c}$ we obtain
\[
	|n|\geq n^cM'
\]
The kth power tick yields
\[
	|n|\geq\lim_{k\to\infty}n^c\sqrt[k]{M'}=n^c
\]
which, together with (**), proves $|n|=n^c$.  Thus, all archimedean
absolute values are equivalent.

$\bf{Case\;II}$  Now suppose that \(|\cdot|\) is non-archimedean.  For
the theorem to hold true, it makes sense that \(|\cdot|\) is
equivalent to \(|\cdot|_p\).  As a result of $|\cdot|$ being 
non-archimedean, the set \(\{n\in\mb N: |n|<1\}\) is nonempty. 
Let $p$ be its minimum element.  We claim that $p$ is a prime
number.  In fact, $p\neq 1$.  If $p=ab$ for some $a,b\in\mb N$,
$a<p,b<p$, then $|a|=|b|=1$, so $|p|=|ab|=1$, a contradiction. 
Next, we show that $|q|=1$ for any $q\in\mb N$ that is not
divisible by $p$. By the division algorithm $q=ap+r$ where 
$a\in\{0,1,2,\dots\}$ and $1\leq r<p$.  Then $|r|=1$ and
$|ap|=|a|*|p|\leq|p|<1$.  By the strong triangle inequality,
$1=|r|\leq\max\{|ap+r|,|-ap|\}=\max\{|q|,|ap|\}=|q|.$  So $|q|\geq
1$, i.e. $|q|=1$.  It follows that for each natural number $n$,
\[
	|n|=|p|^k
\]
where $k$ is the number of factors $p$ of $n$.  We see that for
each $n\in\mb N$
\[
	|n|=|n|_p^c
\]
where $c=-\log|p|(\log(p))^{-1}$.  It follows easily that 
\[
	|x|=|x|_p^c\;\;\;x\in\mb Q
\]
Therefore, all absolute values on $\mb Q$ are equivalent to either
the real absolute value or the p-adic norm.
\end{proof}

Ostrowski's theorem tells us that we can only perform arithmetic 
in \(\mb Q\) with two types of absolute values: the real absolute 
value and the p-adic absolute value.  This is important to us because 
it tells us that even if we created another absolute value and tried to
use it to extend the rational numbers, we would get a set of numbers
equivalent to either the set of real numbers of the p-adic numbers.
Furthermore, one useful result of Ostroski's theorem is that we have a 
way to relate all of the absolute values on $\mb Q$.

\begin{theorem}[Product Formula]
For any $x\in\mb Q^x$, we have
\[
\prod_{p\leq\infty}|x|_p=1
\]
where $p\leq\infty$ means that we take the product over all the primes
of $\mb Q$ including the prime at infinity (which is the real absolute value, 
$|\cdot|_\infty$).
\end{theorem}
\begin{proof}
Let \(x\in\mb Q\).  By the fundamental
theorem of arithmetic, we can decompose a positive $x$
into \(x=p_1^{a_1}*p_2^{a_2}*\dots*p_k^{a_k}\).  It follows that
\[
\begin{cases}
  |x|_q=1 & \text{if }q\neq p_i\\
  |x|_p=p^{-a_i} & \text{for }i=1,2,\dots,k\\
  |x|_\infty=p_1^{a_1}*p_2^{a_2}*\dots*p_k^{a_k}
\end{cases}
\]
Then as a result, we have
\[
  \prod_{p\leq\infty}|x|_p=(p^{-a_1})*(p^{-a_2})*\dots
  *(p_1^{a_1}*p_2^{a_2}*\dots*p_k^{a_k})=1
\]
\end{proof}

The product formula has many applications (many of which are
beyond the scope of this paper) including uses in the theory of
heights on algebraic varieties and the application of Hasse's
local-global principle.

\section{Exploring $\mb Q_p$}

In this section, we will discuss the basic arithmetic operations
on the p-adic numbers.  Through this, we will also expose some
interesting properties of these number systems and compare them to
the real numbers.

\subsection{Writing p-adic Numbers}

Up until now, we have focused on the p-adic numbers as a system as a
whole rather than as individual numbers. We take the time now to
formally define the elements of $\mb Q_p$.  We recall that a 
p-adic number is a number that is written in base p (where p is a
prime number) rather then in the more commonly used 10-ary or decimal
system.  As such, we write the general p-adic numbers as a p-adic expansion;
that is,
\[
	\sum_{k=-n}^\infty a_kp^k=
    a_np^{-n}+a_{n+1}p^{-n+1}+\dots+a_0+a_1p+a_2p^2+\dots,\;\;\;
    a_k\in\{0,1,2,\dots,p-1\}
\]

As a result of the way we construct the Cauchy sequences that form the 
p-adic numbers, we also have the properties that $x_n-x_{n-1}=a_np^n$,
where $x_n$ is the partial sum of the first $n$ elements of the p-adic
expansion, and $x_n\equiv x_{n-1}\pmod {p^n}$.  From the definition of
congruence modulo $p^{n}$, it turns out that we can recover coefficients 
of our p-adic expansion if we have the partial sums.  We know that 
\[
  p^{n}|x_n-x_{n-1}\Rightarrow x_n-x_{n-1}=cp^n\Rightarrow
  c=a_n=\frac{x_n-x_{n-1}}{p^{n}}
\]

Note that in the p-adic expansion, when $k$ is a negative number, 
these terms form the denominator of a rational number.  Alternatively, 
these are equivalent to decimal places of the decimal number.  If
the number is in $\mb Z_p$, then there will be no negative indices.  
Additionally, if we are writing a number in $\mb Q_p$, then it must have
a finite number of negative indices.  We also notice that, in order to 
write a p-adic number in summation form, it looks backwards to how we 
normally write numbers.  As an alternative to writing a p-adic number as we
normally would, we can also express the number in p-ary form.
\[
	x=(\dots a_3a_2a_1a_0.a_{-1}a_{-2}\dots)_p,\;\;\;
    a_i\in\{0,1,\dots,p-1\}\forall\;i\in\mb Z
\]
The final $p$ at the end of the chain of numbers just notes that we
are writing the number in base-p.  Throughout this section, we will
write p-adic numbers in both forms to help adjust to the backwards
notation of a p-adic number.

To make sure that we can confidently read a p-adic number, let's 
convert a few numbers from \(\mb Q\) to $\mb Q_p$.  As an example, we 
will convert 328 to its 5-adic form.  We observe that
\[
	328=2*5^3+3*5^2+3=(2303)_5
\]
We arrived at this conclusion by continuously dividing the number 328 
by 5.  The remainder that we get by dividing by 5 (which must be less 
than 5 by the division algorithm) is the first digit of the p-adic
number.  It is the coefficient multiplied by $p^0$.  In p-ary form, it
is the first digit on the right.

\begin{center}
  \begin{tabular}{|c|c|c|}
    \hline
      Decimal & 5-ary & 5-adic\\
    \hline
      $328=3+5(65)$ & $(3)_5$ & $3p^0$\\
      $65=5(13)$ & $(03)_5$ & $3p^0+0p$\\
      $13=3+5(2)$ & $(303)_5$ & $3p^0+0p+3p^2$\\
      $2=2+5(0)$ & $(2303)_5$ & $3p^0+0p+3p^2+2p^3$\\
    \hline
  \end{tabular}
\end{center}

While the process is a little longer for rational numbers, the idea is 
still the same.  The main difference is that we have two integers to
convert rather than one.  Let $\frac{a}{b}\in\mb Q$.  If $\frac{a}{b}$
is not a proper fraction, make it proper and convert the integer part as
shown above.  If $b$ is divisible by $p$, than the first digit's place
will be whatever power of $p$ most divides the denominator, in other
words $v_p(b)$.  Once $p^{v_p(b)}$ has been factored out of $b$ (so
$p\nmid b$) expand the fraction by the same division algorithm as shown
as above.  Two examples of this process are below.

\begin{example}
In this example, we convert $\frac{242}{25}$ into its 5-adic form.  
Note that our number is not proper, so we write 
$\frac{242}{25}=9\frac{17}{25}$.  Furthermore, $25=5^2$, so we know that
our first digit will be in the $5^{-2}$ spot.  If we factor out
$5^{-2}$, we get $\frac{17}{1}=17$.  To find the 5-adic form of our
number, we need to convert $9$ and $17$ and add the results together.
\begin{center}
  \begin{tabular}{|c|c|c|}
    \hline
      Decimal & 5-ary & 5-adic\\
    \hline
      $9=4+5(1)$ & $(4)_5$ & $4p^0$\\
      $1=5(0)+1$ & $(14)_5$ & $4p^0+1p^1$\\
    \hline
  \end{tabular}
  \qquad
  \begin{tabular}{|c|c|c|}
    \hline
      Decimal & 5-ary & 5-adic\\
    \hline
      $17=2+5(3)$ & $(2)_5$ & $2p^{-2}$\\
      $3=5(0)+3$ & $(32)_5$ & $2p^{-2}+3p^{-1}$\\
    \hline
  \end{tabular}
\end{center}
Now we see that 
\[
  9\frac{17}{25}=\frac{225+17}{25}=2p^{-2}+3p^{-1}+4p^0+1p^1=(14.32)_5
\]  
\end{example}

\begin{example}
In this example, we will convert $\frac{24}{17}$ into its 3-adic form. 
Since this number is both in proper form and does not have a denominator
divisible by 3, we can just proceed into the division algorithm.
\begin{center}
  \begin{tabular}{|c|c|c|}
    \hline
      Decimal & 3-ary & 3-adic\\
    \hline
      $\frac{24}{17}=3(\frac{8}{17})$ & $(0)_3$ & $0p^0$\\
      $\frac{8}{17}=1+3(-3/17)$ & $(10)_3$ & $1p$\\
      $\frac{-3}{17}=3(-1/17)$ & $(010)_3$ & $p+0p^2$\\
      $\frac{-1}{17}=1+3(-6/17)$ & $(1010)_3$ & $p+1p^3$\\
      $\frac{-6}{17}=3(-2/17)$ & $(01010)_3$ & $p+p^3+0p^4$\\
      $\frac{-2}{17}=2+3(-12/17)$ & $(201010)_3$ & $p+p^3+2p^5$\\
      $\frac{-12}{17}=3(-4/17)$ & $(0201010)_3$ & $p+p^3+2p^5+0p^6$\\
      $\frac{-4}{17}=1+3(-7/17)$ & $(10201010)_3$ & $p+p^3+2p^5+1p^7$\\
      $\frac{-7}{17}=1+3(-8/17)$ & $(110201010)_3$ & $p+p^3+2p^5+p^7+1p^8$\\
      $\frac{-8}{17}=2+3(-14/17)$ & $(2110201010)_3$ & $p+p^3+2p^5+p^7+p^8+2p^9$\\
    \hline
  \end{tabular}
\end{center}
If we were to continue our calculations, we would eventually see that
the digits start to repeat.  We completely write the number as
\[
	\frac{24}{17}=(...\overline{0112021221102010}10)_3=
    p+p^3+2p^5+p^7+p^8+2p^9+2p^{10}+p^{11}+2p^{12}+2p^{14}+p^{16}+
    p^{17}+\dots
\]
The overlined part of our 3-ary number is the period of our number.  In 
other words, if we keep writing out the number to the left, our period 
would keep repeating the pattern.  In the p-adic representation, there
is no universal notation to denote a period.  However, the coefficients
of the period would still cycle with the more terms we write out.
In the decimal system, all rational numbers either have a finite number
of digits to the right of the decimal point, or they have a period and
repeat a pattern of numbers. The same thing happens to rational numbers
when they are converted to a p-adic form, but in reverse.  In p-ary
form, they will have a finite number of digits to the right of the
decimal point, but they can have either finite or a periodic string of
numbers to the left of the decimal point.  In p-adic form, this
translates to a finite number of terms with a negative index and either
a set of finite or periodic terms with positive indices.
\end{example}

As shown in the last couple of examples, small and simply-written
numbers in the decimal system can look odd in a p-adic form.  For
instance, we claim that $-1$ is equivalent to $(...2222222.0)_3$.  To
show this, we will find the distance between $-1$ and
$(...2222222.0)_3$.
\[
\begin{gathered}
  \lim_{n\to\infty}|(2+2*3^{1}+2*3^{2}+2*3^{3}+\dots+2*3^n)-(-1)|_3=\\
  \lim_{n\to\infty}|3+2*3^{1}+2*3^{2}+2*3^{3}+\dots+2*3^n|_3=\\
  \lim_{n\to\infty}|3^{n+1}|_3=|0|_3=3^{-v_3(0)}=3^{-\infty}=0
\end{gathered}
\]

We can also show that, after we talk about arithmetic operations in 
$\mb Q_p$, that $(...2222222.0)_3+(...00000001)_3=0$.

\subsection{Arithmetic Operations}

$\mb Q_p$ is a field which means that all of the arithmetic operations
that work in $\mb Q$ will also work on the p-adic numbers.  The good
news is even though these numbers may appear exotic, the normal
operations are still very simple, albeit a bit tedious.

Addition works the same way that we learned it in elementary school.  We 
line up the numbers of the same power of $p$ in the same column and add 
them.  But as our digits are in $\mb Z/p\mb Z$, they must be in the
set $\{0,1,2,\dots,p-1\}$.  If two numbers sum up to more than $p$, it is
replaced in the sum by its congruence modulo $p$ and the integral part
of the sum is divided by p is added to the next column.

\begin{example}
Below, we add the 3-adic numbers $\dots000000001.0_3$ and
$222222222.0_3$.

\begin{center}
  \begin{tabular}{c}
    $\;\;\;\;$$\dots000000001.0_3$\\
    $+$ $\dots222222222.0_3$\\
    \hline
    $\;\;\;\;\dots333333333.0_3$\\
    $\;\;\;\;=000000000.0_3$
  \end{tabular}
\end{center}
\end{example}

The same method works for subtraction is we use the borrowing principle 
from elementary school.  Furthermore, multiplication and division (in
the sense of multiplying by the inverse of a number) can be performed 
in a similar manner.  We can also use multiplication to find the
negative version of a number (the multiplicative inverse).
\newpage
\begin{example}
Below, we multiply the 5-adic numbers $\frac{1}{3}=\overline{13}2.0_5$ 
and $-1=\dots444444444.0_5$.

\begin{center}
  \begin{tabular}{c}
    $\;\;\;\;$$\dots131313132.0_5$\\
    $\times$ $\dots444444444.0_5$\\
    \hline
    $\dots131313133$\\
    $\dots313131330$\\
    $\dots131313300$\\
    + $\dots\dots\dots\dots$\\
    \hline
    $=\dots313131313.0_5$\\
    $=\overline{13}.0_5$
  \end{tabular}
\end{center}
\end{example}

\subsection{Structure of $\mb Q_p$ and $\mb Z_p$}

In this section, we will discuss some of the algebraic properties and
structures involving the p-adic numbers.  We will introduce the concept 
of a valuation ring, and the units in $\mb Z_p$.

\begin{definition}[Valuation Ring]
Let $K$ be a field.  A subring $\textswab{0}$ of $K$ is called a 
$\bf{valuation\;ring}$ if it has the property that for any $x\in K$, 
we have $x\in\textswab{o}$ or $x^{-1}\in\textswab{o}$.
\end{definition}

\begin{definition}
Let $K$ be a field with a non-archimedean valuation $|\cdot|$.  The subset
\[
	\mathcal{O}=\overline{B}_1(0)=\{x\in K:|x|\leq 1\}\subset K
\]
is called the $\bf{valuation\;ring}$ of $|\cdot|$.  Its subset
\[
	\mathcal{B}=B_1(0)=\{x\in K:|x|<1\}
\]
is an ideal of $\mathcal{O}$ called the $\bf{valuation\;ideal}$. 
Furthermore, the valuation ideal is a maximal ideal in $\mathcal{O}$ and
every element of the complement $\mathcal{O}-\mathcal{B}$ is invertible 
in $\mathcal{O}$.  The quotient ring
\[
	\kappa=\mathcal{O}/\mathcal{B}
\]
is called the $\bf{residue\;field}$ of $|\cdot|$.
\end{definition}

If we consider the field $\mb Q$ equipped with $|\cdot|_p$, then we know the 
following:
\begin{enumerate}
  \item The associated valuation ring is $\mathcal{O}=\mb Z_{p}=
  \{\frac{a}{b}\in\mb Q: p\nmid b\}$.
  \item Its valuation ideal is $\mathcal{B}=p\mb Z_{p}=
  \{\frac{a}{b}\in\mb Q: p\nmid b\text{ and }p|a\}$.
  \item The residue field is $\kappa=\mb F_p$ (the field with p 
  elements).
\end{enumerate}

Before moving on to $\mb Q_p$, we would first like to formally define the 
p-adic integers.

\begin{definition}[p-adic Integer]
The $\bf{ring\;of\;p-adic\;integers}$, is the valuation ring
\[
	\mb Z_p=\{x\in\mb Q_p:|x|_p\leq 1\}.
\]
\end{definition}

Recall that $|x|_p=p^{-v_p(x)}$.  We have seen earlier that an element
of $\mb Q_p$ takes the form of a finite-tailed Laurent series.  That is,
the p-adic expansion has a finite number of digits with a negative
index.  Consider the p-adic number 
$x=\dots+a_{-2}p^{-2}+a_{-1}p^{-1}+a_0p^0+a_1p^1+a_2p^2+\dots$ where 
$a_{-1}\not\equiv 0\pmod p$.  Thus, $p^{-1}$ can be factored out of $x$.
So $|x|_p=p^{-v_p(x)}=p^{1}>1$.  It stands to reason that, if we have
any negative indices with coefficients not equal to 0 in our p-adic
expansion, our p-adic absolute value will give us a value greater than
one.  So any p-adic integer, in order to satisfy the above definition, 
will consist of nonnegative powers of p.

If we consider the valuation ideal of $\mb Z_p$, we can determine the
units of the p-adic integers.

\begin{definition}[Unit]
Let $R$ be a ring with unity.  An element $a\in R$ is a $\bf{unit}$ if 
$a$ has a multiplicative inverse in $R$.  That is, $a^{-1}\in R$ such 
that $a*a^{-1}=1_R=a^{-1}*1$.
\end{definition}

One fact about the ring of p-adic integers is that they form a special
type of ring known as a local ring.

\begin{definition}[Local Ring]
A $\bf{local\;ring}$ is a ring that contains a unique maximal ideal
whose complement consists of invertible elements.
\end{definition}

This tells us the maximal ideal of the p-adic integers, $p\mb Z_p$, is
unique and it does not contain any units of $\mb Z_p$.  So if the ring
of p-adic integers contains any units, it would be in the complement of 
$p\mb Z_p$; that is, they would be contained in the set
\[
	\{x\in\mb Q_p:|x|_p=1\}
\]

Let's consider a finite p-adic integer $x=a_0+a_1p+a_2p^2+\dots+a_np^n$.  
We write the following:
\[
\begin{gathered}
	|x|_p=1\Rightarrow |a_0+a_1p+a_2p^2+\dots+a_np^n|_p=1\\
    \Rightarrow \frac{1}{p^{v_p(x)}}=1\Rightarrow
    v_p(x)=v_p(a_0+a_1p+a_2p^2+\dots+a_np^n)=0
\end{gathered}
\]
Recall that each coefficient $a_i\in\{0,1,\dots,p-1\}$.  
If $a_0=0$, then we could, at a minimum, factor $p$ out of $x$ which
would mean that $v_p(x)\neq 0$.  Thus, for $x$ to be a unit in $Z_p$, 
$a_0\not\equiv0\pmod p$.  In fact, the relation works both ways, and we 
can restate this as a theorem.

\begin{theorem}
A p-adic integer $x$ is a unit if and only if $a_0\not\equiv0\pmod p$
\end{theorem}

As a result, we have a new way to represent p-adic integers.  A number 
$x\in\mb Z_p$ can be written as $x=up^n$ where $u$ is a unit in 
$\mb Z_p$ and $n\in\mb Z$.  This result can also be extended to 
$\mb Q_p$.

Furthermore, we have shown that $\mb Z_p$ contains elements which do not
have multiplicative inverses in $\mb Z_p$.  This means that $\mb Z_p$
cannot be a field.  However, as $\mb Z_p\subset\mb Q_p$ and $\mb Q_p$ is
a field, then $\mb Z_p$ cannot contain zero divisors.  Thus, it follows 
that $\mb Z_p$ is an integral domain.  But more specifically, we know
that $\mb Z_p$ is a principle ideal domain and, as a result, a unique
factorization domain.

\begin{definition}[Principle Ideal Domain]
A $\bf{principle\;ideal\;domain}$ is an integral domain in which every
ideal is principal, that is, generated by a single element.
\end{definition}

\begin{definition}[Unique Factorization Domain]
A $\bf{unique\;factorization\;domain}$ is an integral domain $R$ 
satisfying the following properties:
\begin{enumerate}
\item Every nonzero element $a\in R$ can be expressed as 
$a=p_1\dots p_n$, where $u$ is a unit and $p_i$ are irreducible.  That
is, $p_i$ cannot be represented as a product of nonunits.
\item If $a$ has another factorization , sat $a=vq_1\dots q_m$ where $v$ 
is a unit and the $q_i$ are irreducible, then $n=m$ and, after 
reordering if necessary, $p_i$ and $q_i$ are associates for each $i$.  
That is $p_i=sq_i$ for some unit $s\in R$.
\end{enumerate}
\end{definition}

\begin{theorem}
Every principle ideal domain is a unique factorization domain.
\end{theorem}

While the implications of these facts are numerous, we choose to end 
this section focusing on the ideals of $\mb Z_p$. 

\begin{proposition}
The ring $\mb Z_p$ is a principle ideal domain.  More precisely, its 
ideals are the principle ideals $\{0\}$ and $p^k\mb Z_p$ for all
$k\in\mb N$.
\end{proposition}
\begin{proof}
Let $I\neq\{0\}$ be an ideal in $p\mb Z_p$, and let $0\neq a\in I$ be an element of maximum absolute value.  We know this is possible because 
$|\cdot|_p$ forms a set of discrete values.  Suppose that $|a|_p=p^{-k}$
for some $k\in\mb N$.  Then $a=up^k$, $u$ is a unit of $\mb Z_p$.  Then 
$p^k=u^{-1}a\subset I$, and hence $(p^k)=p^k\mb Z_p\subset I$.  Conversely, for any $b\in I$, $|b|_p=p^{-w}\leq p^{-k}$. We can write
\[
	b=p^wu'=p^kp^{w-k}u'\in p^k\mb Z_p
\]
Therefore, $I\subset p^k\mb Z_p$, and hence $I=p^k\mb Z_p$.
\end{proof}

\subsection{Algebraically Closed Sets}

In this section, will demonstrate how, like the real numbers, 
$\mb Q_p$ are not algebraically closed.  We will also go over the
extensions of $\mb Q_p$ that allow it to become closed.  First, we 
must define what we mean by algebraically closed.

\begin{definition}[Algebraically Closed]
A field $K$ is $\bf{algebraically\;closed}$ if it contains a root for 
every polynomial in $K[x]$.
\end{definition}

It is well known that $\mb R$ is not algebraically closed.  Consider
the equation
\[
	x^2+1=0
\]
It does not take much to realize that there is no number in $\mb R$
that satisfies $x^2=-1$.  Eventually, mathematicians represented the
solution to this equation as the letter $i$; in other words, $i^2=-1$. 
With the discovery of $i$ came the discovery of the complex numbers
which, as it turns out, is algebraically closed.  In fact, it is the
smallest field that is algebraically closed that contains $\mb R$.

Well, it turns out that $\mb Q_p$ is not algebraically closed.  For
example, in $\mb Q_5$, the equation $x^2=2$ has no solution.  We will
further discuss solving equations in $\mb Q_p$ in the section on
Hensel's Lemma.  

The point that we want to make in this section is that it is not
guaranteed that $\mb Q_p$ is algebraically closed for every prime
number $p$.  So the question is if there is an extension for $\mb Q_p$ 
that is analogous to $\mb C$.  Well, it turns out that such a field 
does exist, and it is denoted $\mb C_p$.  It is beyond the scope of
this paper to form this field or to work inside of it.  However,
modern p-adic analysis takes place in $\mb C_p$ rather than $\mb Z_p$
or $\mb Q_p$ due to its property of closure, so we wish to at least
mention it for the benefit of the reader.

\section{Effects of Absolute Values on Fields}

\subsection{Ultrametric Space}

It has been mentioned several times that an absolute value gives a
notion of size, magnitude, or distance.  In this short discussion,
we will examine the effects of absolute values on a field, and we
start by formalizing the idea of distance as we did in Advanced Calculus of Several Variables.

\begin{definition}[Metric]
A $\bf{metric}$ on a set $X$ is a function $d:X\times X\to\mb R$
having the following properties:
\begin{enumerate}
  \item $d(x,y)\geq 0$ for all $x,y\in X$; equality holds if and 
  only if $x=y$.
  \item $d(x,y)=d(y,x)$ for all $x,y,\in X$
  \item The triangle equality holds for all $x,y\in X$
\end{enumerate}

The function $d(x,y)$ is often called the $\bf{distance}$ between $x$
and $y$.  The set $X$ equipped with a metric $d$ is written as $(X,d)$
and is called a $\bf{Metric\;Space}$.
\end{definition}

When working with absolute values, our metric $d$ is defined by 
\(d(x,y)=|x-y|\).  Furthermore, if a metric $d(x,y)$ satisfies the 
strong triangle inequality (i.e. \(d(x,y)\leq\max\{d(x,z),d(z,y)\}\)),
or in other words, $d(x,y)$ is defined by a non-archimedean
absolute value, then we call $(X,d)$ an $\bf{ultrametric\;space}$.  As
we know, the p-adic absolute value is non-archimedean.  As a result, 
$(\mb Q_p,|\cdot|_p)$ is an ultrametric space.  So as we explore
the idea and properties of an ulatrametric space, all of the results will hold
true for \(\mb Q_p\).

\begin{theorem}
  Let $K$ be a field equipped with $|\cdot|$, a non-archimedean
  absolute value.  If $x,y\in K$ and $|x|\neq|y|$, then
  \[
  	|x+y|=\max\{|x|,|y|\}
  \]
\end{theorem}
\begin{proof}
Without loss of generality, we assume that \(|x|>|y|\).  As a 
result, by the properties of non-archimedean absolute values,
\[
	|x+y|\leq|x|=\max\{|x|,|y|\}
\]
However, note that \(x=(x+y)-y\).  So we have
\[
	|x|\leq\max\{|x+y|,|y|\}
\]
Since \(|x|>|y|\), this can only be true in the case that
\[
	\max\{|x+y|,|y|\}=|x+y|
\]
This gives us \(|x|\leq|x+y|\), and as \(|x+y|<x\), we know that
\(|x+y|=|x|\)
\end{proof}

From this theorem, we get an interesting corollary:

\begin{corollary}
In an ultrametric space, all "triangles" are isosceles.
\end{corollary}
\begin{proof}
Let $x,y,z$ be the three elements of our ultrametric space that form
the vertices of our triangle.  The length of the sides of the triangle
are 
\[
  \begin{gathered}
    d(x,y)=|x-y|\\
    d(y,z)=|y-z|\\
    d(x,z)=|x-z|
  \end{gathered}
\]
Clearly, \((x-y)+(y-z)=(x-z)\).  By our new theorem, we know that
if \(|x-y|\neq|y-z|\), then \(|x-z|=\max\{|x-y|,|y-z|\}\).  Thus, two
sides of our triangle are equal.
\end{proof}

The above theorem demonstrates how geometry in the p-adic numbers is
very different from euclidean geometry in $\mb R$.  However, open and
closed balls are more widely used in topology and analysis.  So at this
point, we are going to explore the geometry of balls in the p-adic
numbers.

\subsection{Balls}

\begin{definition}[Balls]
Let $K$ be a field with an absolute value $|\cdot|$.  Let $a\in K$ and 
$r\in\mb R_+$.  The \(\bf{open\;ball}\) of radius $r$ and center $a$ is the set
\[
  B_r(a)=\{x\in K:d(x,a)<r\}=\{x\in K:|x-a|<r\}
\]
The \(\bf{closed\;ball}\) of radius $r$ and center $a$ is the set
\[
  \overline{B}_r(a)=\{x\in K:d(x,a)\leq r\}=\{x\in K:|x-a|\leq r\}
\]
\end{definition}

\begin{figure}[!h]
  \centering
  \includegraphics[width=0.3\textwidth]{open-and-closed.png}
  \caption{\label{fig:Balls}The open and closed balls on the 2D plane
  with a radius of \(\epsilon\)}
\end{figure}

The open and closed balls are often used for analyzing metric spaces.  
They are used to show if a set is open or closed.  However, in an 
ultrametric space, due to the non-archimedean properties of the 
absolute value being used, they act a bit differently.

\begin{theorem}
Let $K$ be a field with a non-archimedean absolute value $|\cdot|$.
\begin{enumerate}[label=\Roman*)]
  \item If $b\in B_r(a)$, then $B_r(a)=B_r(b)$; in other words, every
  point that is contained in a open ball is the center of that ball.
  \item If $b\in \overline{B}_r(a)$, then 
  $\overline{B}_r(a)=\overline{B}_r(b)$; in other words, every point
  that is contained in a closed ball is the center of that ball.
  \item The set $B_r(a)$ is clopen, that is, it is both open and
  closed.
  \item If $r\neq 0$, the set $\overline{B}_r(a)$ is clopen.
  \item If $a,b\in K$ and $r,s\in\mb R_+^x$, we have
  $B_r(a)\cap B_s(b)$ if and only if 
  $B_r(a)\subset B_s(b)$ or
  $B_r(a)\supset B_s(b)$; in other words, any two
  open balls are either disjoint or contained in one another.
  \item If $a,b\in K$ and $r,s\in\mb R_+^x$, we have
  $\overline{B}_r(a)\cap\overline{B}_s(b)$ if and only if 
  $\overline{B}_r(a)\subset\overline{B}_s(b)$ or
  $\overline{B}_r(a)\supset\overline{B}_s(b)$; in other words, any two
  closed balls are either disjoint or contained in one another.
\end{enumerate}
\end{theorem}

\begin{figure}[!h]
  \centering
  \includegraphics[width=0.7\textwidth]{arbol.jpg}
  \caption{\label{fig:arbol}Visualization of the balls in
  $Q_3$}
\end{figure}

\subsection{Topological Properties}

For the rest of this section, we wish to topologically describe $\mb Q_p$
and $\mb Z_p$.  We will not prove anything; we only wish to leave the 
reader with definitions and theorems to describe properties of our new 
numbers - several of which also hold true for $\mb R$ as we discovered in 
Advanced Calculus of a Single Variable.

\begin{theorem}
In a field $K$ with a non-archimedean norm $|\cdot|$, the
connected component of any point $x\in K$ is the set $\{x\}$
consisting of only that point.  In other words, $\mb Q_p$ is totally 
disconnected.
\end{theorem}

\begin{corollary}
$\mb Q_p$ is complete and separable.
\end{corollary}

The idea of connectedness is that, if we chose any two numbers of a set 
on a graph or plane, we could draw a path between those two numbers 
without leaving the set.  In more formal terms, a connected space is one
that cannot be broken up into a collection of disjoint, open sets.

For example, consider the real numbers which form a connected set.  If we 
looked at any open interval $(a,b)\subset\mb R$ such that $c\in(a,b)$, 
then we know that $(a,b)\neq(a,c)\cup(c,b)$ because $c\in(a,c)\cup(c,b)$.
However, the same thing cannot be said for $\mb Q_p$ because it is 
totally disconnected.  Formally, $\mb Q_p$ forms a Hausdorff space - a 
space in which distinct points have disjoint neighborhoods.

\begin{definition}[Compact Set]
A set is $\bf{compact}$ if every sequence in the set has a subsequence
that converges inside of the set.  Furthermore, a compact set is closed and bounded.
\end{definition}

\begin{theorem}
$\mb Z_p$ is compact.
\end{theorem}

\begin{corollary}
$\mb Z_p$ is complete.
\end{corollary}

\begin{theorem}
$\mb Z$ is dense in $\mb Z_p$.
\end{theorem}

\begin{theorem}
$\mb Q_p$ is locally compact; $\mb Q$ is dense in $\mb Q_p$.
\end{theorem}

It is interesting to note that $\mb Q_p$ is not compact.  Rather, it is 
locally compact.  What this means is that every point in $\mb Q_p$ has a
neighborhood that is compact.  So while $\mb Q_p$ is not compact, for any
p-adic number, there exists a subset of $\mb Q_p$ containing that number
that is compact.  This makes sense as $\mb Q_p$ is totally disconnected. 
It follows from this fact
that the neighborhood for any $x\in\mb Q_p$ can only
be $\{x\}$.  Any sequence formed in a set of one element is a constant
sequence that clearly converges to itself, which makes $\{x\}$ compact.

\section{Applications of $\mb Q_p$}

\subsection{Hensel's Lemma}

\begin{theorem}[Hensel's Lemma]
Let $F(X)=a_0+a_1X+a_2X^2+\dots+a_nX^n$ be a polynomial in $\mb Z_p[X]$
- the ring of polynomials whose coefficients are in $\mb Z_p$.  Suppose
that there exists a p-adic integer $\alpha_1\in\mb Z_p$ such that
\[
	F(\alpha_1)\equiv0\pmod {p\mb Z_p}
\]
and 
\[
	F'(\alpha_1)\not\equiv0\pmod {p\mb Z_p}
\]
where $F'(X)$ is the derivative of $F(X)$.  Then there exists a
p-adic integer $\alpha\in\mb Z_p$ such that 
$\alpha\equiv\alpha_1\pmod {p\mb Z_p}$ and $F(\alpha)=0$.
\end{theorem}

\begin{remark}
Before we move on, we wish to mention that when we are referring to the
derivative of a polynomial in $\mb Z[p]$, we mean the power rule for
derivatives in $\mb R[x]$.  The formal definition of a derivative is
harder to execute due to the fact that $\mb Q_p$ is a disconnected
space.
\end{remark}

Hensel's lemma is one of the most import tools in elementary p-adic 
analysis.  While it does not work for every equation, it does work for
many simple ones.  It allows us to not only determine if a solution
exists, but to also determine an exact solution if it exists by following 
the process of its proof.  This process is actually similar to Newton's
Method which we learned about in Calculus I.

Before we prove Hensel's Lemma, we remind ourselves of Newton's method so 
that we can observe the parallels between it and the proof.  Let $x_n$ be
an approximate solution to $f(x)=0$.  If $f'(x_n)\neq 0$, then the next
approximation $x_{n+1}$ is given by
\[
	x_{n+1}=x_n-\frac{f(x_n)}{f'(x_n)}
\]

\begin{proof}
Our goal is to construct a Cauchy sequence whose limit is the root 
$\alpha$ that satisfies the hypothesis for the lemma.  More specifically, 
we will form a sequence of integers $\alpha_1,\alpha_2,\dots,\alpha_n$ such 
that
\[
  \begin{gathered}
    F(\alpha_n)\equiv0\pmod{p^n}\\
    \alpha_{n}\equiv\alpha_{n+1}\pmod{p^n}
  \end{gathered}
\]
To start, we assume that an $\alpha_1$ exists.  To find $\alpha_2$ we 
note that by the above conditions,
\[
	\alpha_1\equiv\alpha_2\pmod{p}\Rightarrow
    \alpha_2\equiv\alpha_1\pmod{p}\Rightarrow
    \alpha_2=\alpha_1+b_1p,
    \;\;\;\;\;b_1\in\mb Z_p
\]
by the definition of congruence modulo p.  By plugging the above into our 
function and expanding it into a Taylor Series, we get
\[
	F(\alpha_2)=F(\alpha_1+b_1p)=F(\alpha_1)+F'(\alpha_1)b_1p+\dots
    \equiv F(\alpha_1)+F'(\alpha_1)b_1p\pmod{p^2}
\]
It follows that, in order to prove show we can find $\alpha_2$, we must 
solve
\[
  F(\alpha_1)+F'(\alpha_1)b_1p\equiv 0\pmod{p^2}
\]
By our hypothesis that $F(\alpha_1)\equiv0\pmod{p}$, we know that 
$F(\alpha_1)=px$ for some $x$.  By substitution,
\[
	px+F'(\alpha_1)b_1p\equiv0\pmod{p^2}\Rightarrow x+F'(\alpha_1)b_1
    \equiv0\pmod p
\]
From the condition that $F'(\alpha_1)\not\equiv0\pmod p$, we know that
$p$ does not divide $F'(\alpha_1)$.  As a result, we know that
$F'(\alpha_1)$ is invertible in $\mb Z_p$.  So we take
\[
	b_1\equiv-x(F'(\alpha_1))^{-1}\pmod{p}
\]
Furthermore, we can choose a $b_1$ in $\mb Z$ such that 
$0\leq b_1\leq p-1$ so that $b_1$ is uniquely determined.  For this 
choice of $b_1$, we set $\alpha_2=\alpha_1+b_1p$ which satisfy the above 
properties.  From here we can uniquely determine successive values of 
$\alpha_n$ and $\alpha_{n+1}$ which constructs our sequence and proves
our theorem.
\end{proof}

For higher values of $n$, the formulas are restated below.
\begin{enumerate}
  \item $F(\alpha_n)=p^nx\pmod{p^{n+1}}$
  \item $x+F'(\alpha_1)b_n\equiv0\pmod p$
  \item $\alpha_{n+1}=\alpha_n+b_np$
\end{enumerate}

Compare the above equations to the process in the proof, and note that
for all values of $n$, the derivative can always be calculated at 
$\alpha_1$.  This is because, if we look at the Taylor series expansion,
\[
	F'(\alpha_n)=F'(\alpha_1)+F''(\alpha_2)b_2p+\dots\equiv 
    F'(\alpha_1)\pmod{p}
\]
So due to the modulo $p$, we can reuse the same value in 2. above for our 
iterations.

\begin{example}
  To help show the process for using Hensel's Lemma, we solve
  \[
  	x^2=7,\;\;\;x\in\mb Q_3
  \]
\end{example}
\begin{solution}
We define our function by $f(x)=x^2-7$ where $x\in\mb Q_3$.  First we
need to solve
\[
	\alpha_1^2\equiv 7\equiv1\pmod 3
\]
which we can clearly solve by $\alpha_1=1,2$.  We choose $\alpha_1=1$ to
start our solution (starting with $\alpha_1=2$ simply gives another
solution).  We then write
\[
  f(\alpha_1)=f(1)=-6\equiv3\pmod{3^2}=3(1)=px
\]
We also note that $f'(x)=2x\Rightarrow f'(1)\equiv2\pmod 3$.  It follows 
that
\[
  px+f'(\alpha_1)b_1p\equiv0\pmod{p^2}\Rightarrow x+f'(\alpha_1)b_1
  \equiv0\pmod p\Rightarrow 1+(2)b_1\equiv0\pmod 3\Rightarrow b_1=1
\]
So by our equation for $\alpha_2$, we write 
\[
  \alpha_2=\alpha_1+b_1p=1+1*3=4
\]
Repeating for our next iteration, we find that 
\[
  \begin{gathered}
    f'(\alpha_2)\equiv1\pmod 3,\\
    f(\alpha_2)=16-7\equiv9(1)\equiv p^2x\pmod{3^3}\Rightarrow 
    x\equiv1\pmod 3,\\
    x+f'(\alpha_2)b_2\equiv0\pmod 3\Rightarrow 
    1+b_2\equiv0\pmod 3\Rightarrow
    b_2=2,\\
    \alpha_3=\alpha_2+b_2p=4+2*3=10
  \end{gathered}
\]
Continuing this process allows us to solve uniquely for $\alpha_n$ 
provided that the derivative at $\alpha_n$ is nonzero.  After continuing 
for four iterations and reducing the coefficients modulo 3, we get
\[
  \alpha_4=1+1*3+1*3^2+0*3^3+2*3^4+\dots
\]
\end{solution}

As a result of Hensel's Lemma, we can now determine several properties of 
p-adic numbers such as determining what numbers are quadratic residues, 
square roots, squares, and roots of unity - types of numbers that we 
learning about in Intro to Complex Analysis and Number Theory.  First, we 
will see how to determine if a p-adic number is a quadratic residue.

\begin{theorem}
A polynomial with integer coefficients (an element of $\mb Z[X]$) has a
root in $\mb Z_p$ if and only if it has a root modulo $p^k$ for some
integer $k\geq 1$.
\end{theorem}

\begin{definition}[Quadratic Residue]
Let $a,x,m\in\mb Z$.  We say that $x$ is a $\bf{quadratic\;residue}$ 
with respect to m if it satisfies the equation
\[
  x^2\equiv a\pmod m
\]
\end{definition}

\begin{proposition}
Let $a\in \mb Z$ and $p\neq 2$ be prime such that $p\nmid a$.  
$a$ has a square root in $\mb Z_p$ if and only if $a$ is a 
quadratic residue modulo $p$.
\end{proposition}
\begin{proof}
Let $P(x)\equiv x^2-a$. Then $P'(x)=2x$.  Suppose that 
$a$ is a quadratic residue of $p$. Then we know
\[
	a\equiv a_0^2\pmod p,\;\;\;\;\;a_0\in\{0,1,2,\dots,p-1\}
\]
As a result, $P(a_0)=0\pmod p$.  However, $P'(a_0)=2a_0\not\equiv0\pmod p$ because 
$p\nmid a_0\Rightarrow \gcd(p,a_0)=1$.  Hence, by Hensel's lemma, $P(x)$ has a solution
in $\mb Z_p$ which means that $a$ has a square root in $\mb Z_p$.$\\$
Conversely, if $a$ is a non-quadratic residue with respect to $p$, then by the previous 
theorem, it has no square root in $\mb Z_p$.
\end{proof}

As a result of this proposition, we can see that the imaginary number,
$\sqrt{-1}$, is 
an element of $\mb Z_5$.  This is because 
$-1=4-5\equiv 4\pmod 5\equiv 2^2\pmod 5$. 
Since $-1\equiv 4\pmod 5$ and $4$ is a quadratic residue with respect to $5$, 
then by our proposition, $-1$ has a square root in $\mb Z_p$.  Now, we will examine the roots of unity of $\mb Q_p$.

\begin{definition}[Primitive Root Modulo n]
Let $g,a,n\in\mb Z$.  We say that $g$ is a $\bf{primitive\;root\;modulo\;n}$
if, for every integer $a$ relatively prime to $n$, $a$ is congruent to
some power of $g$.  In symbols, $a\equiv g^k\pmod n$ where $\gcd(a,n)=1$ 
and $k\in\mb Z$.
\end{definition}

\begin{definition}[Roots of Unity]
We call $\rho$ an nth root of unity if $\rho^n=1$.  Additionally, we say that 
$\rho$ is a primitive $nth$ root of unity if $\rho^n=1$ such that 
$\rho^k\neq 1$ for all $k<n$.
\end{definition}

Before going over the next proposition, we must first recall the idea
of groups and cyclic groups that we learned in Modern Algebra.

\begin{definition}[Group]
A $\bf{group}$ is a set of numbers $G$ under a single binary operation, 
denoted by $*$, such that, for all $a,b,c\in G$, it fulfills the 
following properties:
\begin{enumerate}
  \item $a*b\in G$ and $b*a\in G$
  \item There exists a unique element $e\in G$ such that $e*a=a=a*e$
  \item For any $a\in G$, there exists a unique element $a^{-1}\in G$ such that
  $a*a^{-1}=a^{-1}*a=e$.
\end{enumerate}
\end{definition}

\begin{definition}[Subgroup]
Let $G$ be group. Then a subset $H\subseteq G$ is a $\bf{subgroup}$ of 
$G$, written $H\leq G$, if $H$ forms a group under the same binary 
operation of $G$.
\end{definition}

\begin{definition}[Cyclic Group Generated by a]
Let $a$ be an element of group $G$.  We write $\<a\>$ to be the 
$\bf{cyclic\;group}$ generated by $a$ defined by $\<a\>=\{a^n:n\in\mb Z\}$. 
Furthermore, $\<a\>\leq G$.
\end{definition}

\begin{definition}[Order of a Group]
Let $G$ be a group.  The $\bf{order}$ of $G$, written as $|G|$, is the 
number of elements in the group.
\end{definition}

\begin{proposition}
For any prime $p$ and any coprime positive integer $m$, there 
exists a primitive mth root of unity in $\mb Q_p$ if and only 
if $m|(p-1)$.  In the latter case, every mth root of unity is 
also a $(p-1)$th root of unity.  The set of $(p-1)$th roots of 
unity is a cyclic subgroup of $\mb Z_p^x$, the units of $\mb Z_p$, 
of order $(p-1)$
\end{proposition}
\begin{proof}
Let $m|(p-1)$; then $p-1=km$ for $k\geq 1$, and therefore any 
$m$th root of $1$ is also a $(p-1)$th root of $1$.  Let
\[
  f(x)=x^{p-1}-1,\;\;f'(x)=(p-1)x^{p-2}
\]
Take $x_0\in\mb Z_p^x$ to be any rational integer satisfying
$1\leq x_o\leq p-1$. Then
\[
  f(x_0)\equiv 0\pmod p\;\;\;\;f'(x_0)\not\equiv 0\pmod p
\]
since $|f'(x_0)|_p=1$, and Hensel' lemma applies, giving exactly 
$p-1$ solutions, which are the $(p-1)$th roots of 1.  The
first digits of these roots are $1,2.\dots,p-1$.  Conversely, is 
$\alpha\in\mb Q_p$ is an $m$th root of 1, $\alpha^m\equiv 1\pmod p$;
hence $m$ divies $p-1$, the order of $(\mb Z/p\mb Z)^x$.  Since a 
polynomial with coefficients in a field can only have as many roots
as its degree, the polynomial $x^{p-1}-1$ cannot have more than 
$p-1$ roots, and these roots must be the roots of unity in $\mb Q_p$.
It is clear that these roots of unity form a group under multiplcation.
Finally, since any finite group of the multiplicative group of any 
field is cyclic, the group of $(p-1)$th roots of unity is a cyclic
subgroup of $\mb Z_p^x$ of order $p-1$.
\end{proof}

\subsection{Local Analysis and Hasse's Principle}

Throughout this paper, we have been looking at $\mb Q_p$ as a 
whole.  In this section, we wish to emphasize that $\mb Q_p$ is
different for every value of p.  With Hensel's lemma, we solved
polynomials with p-adic integer coefficients.  Things get a little
harder when we try to solve a p-adic equation with rational
coefficients.

However, there is an important relationship between $\mb Q$ and each 
$\mb Q_p$.  That is the fact that $\mb Q\hookrightarrow\mb Q_p$.  Suppose 
that a polynomial had a root in $\mb Q$.  Then it would be 
obvious that the polynomial would have a root in each $\mb Q_p$.  We want 
to consider the converse.  What if we could know that the root of a 
polynomial existed in every $\mb Q_p$.  That is, what if we knew that the
root existed locally.  That would imply that the root existed in the
rational numbers - that it exists globally.

The whole point of finding roots locally in order to find information 
about the root globally is that it can be easier to find a root locally. 
We have an example of this below.

\begin{proposition}
A number $x\in\mb Q$ is square if and only if it is a square in every 
$\mb Q_p$, $p\leq\infty$.
\end{proposition}
\begin{solution}
For any $x\in\mb Q_p$,  we know from a slight derivation to the product
formula that 
\[
	x=\pm\prod_{p<\infty}p^{v_p(x)}.
\]
If $x$ is square at infinity (in $\mb R$), then it is positive.  If it is
a square at a prime $p$, then $v_p(x)$ is even.  It follows by
writing out the prime factorization that $x$ is a square.
\end{solution}

\begin{definition}[Diophantine Equations]
A $\bf{diophantine\;equation}$ is an equation in which only integer
solutions are allowed.  It can have several variables, and the most
common form is
\[
	X^2+Y^2=Z^2
\]
\end{definition}

The diophantine equations are an excellent example of analysis on a local 
and global scale.  Consider the equation
\[
  X^2+Y^2+Z^2=0
\]
In $\mb R$, it is clear that this equation does not have a non-trivial 
solution (that is, $X=Y=Z=0$).  Thus, the result is the same for $\mb Q$.
Likewise, we can check that for the following equation
\[
  X^2+Y^2-Z^2=0
\]
has a solution in $\mb Q$ (for example, $X=4$, $Y=3$, $Z=5$) which 
implies that each field $\mb Q_p$ also contains a solution.  This 
local-global relationship results in the following principle.

\begin{definition}[Local-Global Principle]
The existence or non-existence of solutions in $\mb Q$ (global solutions)
of a diophantine equation can be detected by studying, for each
$p\leq\infty$, the solutions of the equation in $\mb Q_p$.
\end{definition}

If it was not already clear, the Local-Global Principle is a bit  
vague.  While it is not a theorem, it does provide a plan to study,
analyze, and solve certain types of diophantine equations.  And as a
result, we have several findings due to local analysis such as the
following major theorem.

\begin{theorem}[Hasse-Minkowski]
Let
\[
	F(X_1,X_2,\dots,X_n)\in\mb Q[X_1,X_2,\dots,X_n]
\]
be a quadratic form (that is, homogenious polynomial of degree 2 in $n$ 
variables).  The equation
\[
	F(X_1,X_2,\dots,X_n)=0
\]
has non-trivial solutions in $\mb Q$ if and only if it has non-trivial
solutions in $\mb Q_p$ for each $p\leq\infty$.
\end{theorem}

The proof of this theorem requires many ideas in the study of 
quadratic forms,
so we will not prove it here.  However, the theorem is a major 
tool in local
analysis, and is of great use in modern number theory and 
diophantine analysis.

\subsection{Fermat's Last Theorem}

\begin{theorem}[Fermat's Last Theorem]
For all natural numbers $n>2$, the equation
\[
	X^n+Y^n=Z^n
\]
has no integral solutions other than the trivial solution where one of
the integers X,Y,Z is equal to 0.
\end{theorem}

Originally conjectured by Pierre de Fermat around 1637, it was not until 
1995 when the theorem was formally and completely proven by Andrew Wiles. 
In the 350 years that it took to prove, this theorem was one of the
greatest mysteries and challenges of number theory.  It took several
discoveries and research by many individuals in areas including
modular forms, elliptic curves, and Galois theory before Wiles could
connect the dots.

Throughout the complex proof, p-adic numbers were used constantly to help
prove cases of Fermat's last theorem when $n$ was a prime number $p$ or
$p^m$ for any $m\in\mb Z$.  Properties of the p-adic norm and the numbers
in general allowed several shortcuts and gave us valuable information
when the equation was calculated modulo $p^m$.  Unfortunately, it would
take an entire paper just to explain all the nuances of the proof due to
all the background information from the different fields that it takes to
understand it.  While we will not be discussing the proof, it is worth
mentioning how the p-adic number played such an integral part in proving
one of mathematics' most famous theorems.

\section{References}

\begin{enumerate}
  \item Katok, Svetlana. 2007. p-adic Analysis Compared with Real. Providence, Rhode
  Island: American Mathematical Society.
  \item Gouvea, Fernando. 1997. p-adic Number. Switzerland: Springer International
  Publishing.
  \item Lang, Serge. 2002. Algebra. Switzerland: Springer International Publishing.
  \item Schikhof, Wilhelmus. 2007. Ultrametric Calculus: An Introduction. Cambridge,
  United Kingdom: Cambridge University Press.
  \item Mahler, Kurt. 1973. An Introduction to p-adic Numbers and their Functions. 
  Cambridge, United Kingdom: Cambridge University Press.
  \item Ribenboim, Paulo. 2000. Fermat's Last Theorem for Amateurs. Switzerland:
  Springer International Publishing.
  \item Ash, Robert. 2006. Basic Abstract Algebra for Graduate Students and Advanced
  Undergraduates. Mineola, New York: Dover Publications.
  \item Munkres, James. 2015. Topology. London, United Kingdom: Pearson.
  \item Hungerford, Thomas. 2012. Abstract Algebra: An Introduction. Independence,
  Kentucky: Cengage Learning.
  \item Gallian, Joseph. 2012. Contemporary Abstract Algebra. Independence, Kentucky:
  Cengage Learning.
  \item Heath-Brown, Roger, et al. 2008.  An Introduction to the Theory of Numbers.
  Oxford, United Kingdom: Oxford University Press.
\end{enumerate}.

\end{document}
